
% R00 17.ABR.2018 - Edson Bispo Ferreira.
%
%-------------------------------------------------------------------------

\documentclass[
	12pt,				
	oneside,			
	a4paper,			
	english,			
	brazil				
	]{abntex2ppgsi}

% ---
% Pacotes básicos 

\usepackage[utf8]{inputenc}		% Codificacao do documento (conversão automática dos acentos)
\usepackage{lastpage}			% Usado pela Ficha catalográfica
\usepackage{indentfirst}		      % Indenta o primeiro parágrafo de cada seção.
\usepackage{color}				% Controle das cores
\usepackage{graphicx}		 	% Inclusão de gráficos
\usepackage{microtype} 			% para melhorias de justificação
\usepackage{pdfpages}            %para incluir pdf
\usepackage{algorithm}			%para ilustrações do tipo algoritmo
\usepackage{mdwlist}			%para itens com espaço padrão da abnt
\usepackage[noend]{algpseudocode}			%para ilustrações do tipo algoritmo
\usepackage{quoting}			% pacote de citações (quotes)
\usepackage{setspace}			% pacote singlespace, para citações
\usepackage[Symbol]{upgreek}    %pacote para letras gregas maiúsculas
\usepackage{siunitx}		
\usepackage{multirow}	
		
% ---
% Pacotes de citações
% ---
\usepackage[brazilian,hyperpageref]{backref}	 % Paginas com as citações na bibl
\usepackage[alf]{abntex2cite}	% Citações padrão ABNT

% --- 
% CONFIGURAÇÕES DE PACOTES
% --- 

% ---
% Configurações do pacote backref
% Usado sem a opção hyperpageref de backref
%\renewcommand{\backrefpagesname}{Citado na(s) página(s):~}
% Texto padrão antes do número das páginas
%\renewcommand{\backref}{}
% Define os textos da citação
%\renewcommand*{\backrefalt}[4]{
%	\ifcase #1 %
%		Nenhuma citação no texto.%
%	\or
%		Citado na página #2.%
%	\else
%		Citado #1 vezes nas páginas #2.%
%	\fi}%
% ---

% ---
% Informações de dados para CAPA e FOLHA DE ROSTO
% ---

%-------------------------------------------------------------------------
% Comentário adicional do PPgSI - Informações sobre o ``instituicao'':
%
% Não mexer. Deixar exatamente como está.
%
%-------------------------------------------------------------------------
\instituicao{
	INSTITUTO FEDERAL DE EDUCAÇÃO, CIÊNCIA E TECNOLOGIA 
      \par
      DE SÃO PAULO
	\par
	CAMPUS SÃO PAULO
	\par
	MESTRADO ACADÊMICO EM ENGENHARIA MECÂNICA}


%
%-------------------------------------------------------------------------
\titulo{Análise de falha em rolamentos por análise de vibração, com utilização de piezoelétrico polimérico}


%-------------------------------------------------------------------------
\autor{\uppercase{GIOVANI FONSECA RAVAGNANI DISPERATI}}

%-------------------------------------------------------------------------
\local{São Paulo}

%-------------------------------------------------------------------------
\data{2019}


%-------------------------------------------------------------------------
\orientador{Prof. Dr. Wilson Carlos da Silva Junior}

\tipotrabalho{Qualificação (Mestrado)}

\preambulo{
%-------------------------------------------------------------------------
%  
%
%
% 
%
%-------------------------------------------------------------------------
. \newline \newline \newline 
%-------------------------------------------------------------------------
Proposta de qualificação apresentada ao Departamento de Engenharia Mecânica do Instituto Federal de Educação, Ciência e Tecnologia de São Paulo, Campus São Paulo como requisito parcial para a obtenção do título de Mestre em Ciências em Engenharia Mecânica. 
%
\newline \newline Área de concentração: Engenharia Mecânica
%-------------------------------------------------------------------------
% 
% 
%
% 
% 
% 
% 
%
%-------------------------------------------------------------------------
\newline \newline \newline .}

\definecolor{blue}{RGB}{41,5,195}

% informações do PDF
\makeatletter
\hypersetup{
     	%pagebackref=true,
		pdftitle={\@title}, 
		pdfauthor={\@author},
    	pdfsubject={\imprimirpreambulo},
	    pdfcreator={LaTeX com abnTeX2 adaptado para o ifsp},
		pdfkeywords={abnt}{latex}{abntex}{abntex2}{qualificação de mestrado}{dissertação de mestrado}{ppgsi}, 
		colorlinks=true,       		% false: boxed links; true: colored links
    	linkcolor=black,          	% color of internal links
    	citecolor=black,        		% color of links to bibliography
    	filecolor=black,      		% color of file links
		urlcolor=black,
		bookmarksdepth=4
}
\makeatother
% --- 

% --- 
% Espaçamentos entre linhas e parágrafos 
% --- 

% O tamanho do parágrafo é dado por:
\setlength{\parindent}{1.25cm}

% Controle do espaçamento entre um parágrafo e outro:
\setlength{\parskip}{0cm}  % tente também \onelineskip
\renewcommand{\baselinestretch}{1.5}

% ---
% compila o indice
% ---
\makeindex
% ---

	% Controlar linhas orfas e viuvas
  \clubpenalty10000
  \widowpenalty10000
  \displaywidowpenalty10000

% ----
% Início do documento
% ----
\begin{document}

% Retira espaço extra obsoleto entre as frases.
\frenchspacing 

% ----------------------------------------------------------
% ELEMENTOS PRÉ-TEXTUAIS
% ----------------------------------------------------------
% \pretextual

% ---
% Capa
% ---

\imprimircapa
% ---

% ---
% Folha de rosto
% ---
\imprimirfolhaderosto
% ---



% ---
% RESUMOS
% ---

% resumo em português
\setlength{\absparsep}{18pt} % ajusta o espaçamento dos parágrafos do resumo
\begin{resumo}

%-------------------------------------------------------------------------

%\begin{flushleft}
%DISPERATI, Giovani Fonseca Ravagnani. \imprimirtitulo. \imprimirdata. \pageref{LastPage} f. Proposta de qualificação (Mestrado em Ciências) %– Departamento de Engenharia Mecânica, Instituto Federal de Educação, Ciência e Tecnologia de São Paulo, São Paulo, 2018.
%\end{flushleft}

A manutenção preditiva é uma área de grande relevância tendo em vista os fatores econômicos e de segurança associados. Nos maquinários, rolamentos são partes imprescindíveis para seu funcionamento. Propõe-se, neste trabalho, o estudo da possibilidade de realização de análise preditiva de falhas em rolamentos com utilização de filmes de Difluoreto de Polivinilideno, PVDF, material que apresenta elevados níveis de piezoeletricidade e propriedades de resistência à corrosão, maleabilidade e alto espectro de frequências, além de baixo custo. Para a condução dos experimentos será fabricada uma bancada de testes e um sistema de aquisição de dados afim de validar a viabilidade de utilização do sensor para a aplicação proposta. 

Palavras-chaves: Manutenção preditiva. Rolamentos. Análise de vibração. PVDF.
\end{resumo}

% resumo em inglês

\begin{resumo}[Abstract]
\begin{otherlanguage*}{english}

Predictive maintenance is an area of great relevance in view of the associated economic and safety factors. In machinery, bearings are essential parts for its operation. It is proposed, in this work, the study of the possibility of performing predictive analysis of failure in bearings using polyvinylidene difluoride films, PVDF, a material with high levels of piezoelectricity and properties of resistance to corrosion, malleability and high spectrum of frequencies, in addition to low cost. For the conduction of the experiments a bench of tests and a data acquisition system will be made.

Keywords: Predictive Maintenance. Bearings. Vibration analysis. PVDF.
\end{otherlanguage*}
\end{resumo}

% ---
% ---
% inserir lista de figuras
% ---
\pdfbookmark[0]{\listfigurename}{lof}
\listoffigures*
\cleardoublepage
% ---
% inserir lista de tabelas
% ---
\pdfbookmark[0]{\listtablename}{lot}
\listoftables*
\cleardoublepage
% ---

% ---
% inserir lista de abreviaturas e siglas
% ---
%-------------------------------------------------------------------------
% Comentário adicional do PPgSI - Informações sobre ``Lista de abreviaturas 
% e siglas'': 
%
% Opcional.
% Uma vez que se deseja usar, é necessário manter padrão e consistência no
% trabalho inteiro.
% Se usar: inserir em ordem alfabética.
%
%-------------------------------------------------------------------------
\begin{siglas}
  \item[PVDF] Fluoreto de Polivinilideno
\end{siglas}
% ---

% ---
% inserir lista de símbolos
% ---

\begin{simbolos}
  \item[$ \infty $] Letra grega Gama
  \item[$ \Lambda $] Lambda
  \item[$ \zeta $] Letra grega minúscula zeta
  \item[$ \in $] Pertence
\end{simbolos}
% ---

% ---
% inserir o sumario
% ---
\pdfbookmark[0]{\contentsname}{toc}
\tableofcontents*
\cleardoublepage
% ---



% ----------------------------------------------------------
% ELEMENTOS TEXTUAIS
% ----------------------------------------------------------
\textual



%-------------------------------------------------------------------------

\chapter{Introdução}
Tem-se vários tipos de polímeros, os de maior interesse tecnológico são os eletroativos que são capazes de realizar a conversão entre energia elétrica e energia mecânica. Em outras palavras, estes polímeros apresentam alteração em seu tamanho e/ou forma quando são sujeitos a um campo elétrico

Alguns  materiais dentro desta família apresentam também outro efeito: geram um sinal elétrico quando sujeitos a uma força. Devido a estas caraterísticas, os polímeros eletroativos são ideais para a aplicação em sensores e atuadores, onde por sua baixa massa especifica, substituem sistemas pesados e complexos, possibilitando projetos de novas aplicações devido a uma mais simples implementação e miniaturização (Bar-Cohen, 2002).

O fluoreto de polivinilideno – PVDF, apresenta características piezo e piroelétricas, que o torna, um excelente elemento para utilização tecnológica, sendo a principal aplicação em sensores, além de sua estabilidade e resistência química e física, combinadas ao fácil processamento deste material. O PVDF apresenta 4 fases cristalinas, sendo $\alpha$, $\beta$, $\gamma$ e $\delta$ (Silva, 2009). A fase beta é de interesse tecnológico por possuir características piezo e piroelétrica.

O PVDF, foi descoberto em 1969 por Kawai, sendo um material novo, sabe-se que as suas propriedades piezeletricidade, após ser mecanicamente tracionado e submetido à ação de um campo elétrico (Wisnieski, 2002).

O PVDF, é flexível, são relativamente de fáceis fabricação e seu custo de fabricação é baixo.  A sua cristalização ocorre em pelo menos quatro distintas fases  cristalinas, denominadas: alfa, beta, gama e delta. É um material semicristalino, de fácil usinagem, múltiplas aplicações e com grande resistência química (Fukada, 2000; Lang, 2006).

\chapter{Objetivos}

\section{\textbf {Geral}}

Em relação aos objetivos do trabalho, visa-se, de forma geral, a aquisição de sinal de vibração, com a utilização de sensor PVDF, para que seja realizada a detecção de falhas em rolamentos em um sistema supervisor com aplicações de manutenção preditiva.

\section{\textbf {Específicos}}

Os objetivos específicos do trabalho podem ser sumarizados em:

\begin{itemize}
	\item Diferenciar um sinal de falha de rolamento de esfera com uma, com as seguintes falhas:1- desgaste da pista do anel interno, 2- desgaste na pista do anel externo e 3- desgaste no elemento rolante.
	\item Determinar um nível de alarme, indicativo de uma situação pré falha.
	\item Encontrar o melhor posicionamento do sensor para captar os sinais de falha.\
	\item Fabricar um sistema de aquisição de dados para o sensor PVDF, quando trabalhando com sinais de falha de rolamento.
\end{itemize}

\chapter{Fundamentação teórica e revisão da Literatura}
Neste capítulo abordamos os conceitos essenciais que embasam este projeto, a saber: os fundamentos da manutenção preditiva, tópico focal de nosso projeto; os fundamentos, aplicações e revisão de falhas de mancais de rolamentos, em particular considerando os que serão abordados nos experimentos práticos que serão conduzidos; os fundamentos de vibração mecânicas, afim de embasar a aquisição de sinais vibracionais para análise preditiva de falhas nos mancais de rolamentos; fundamentos da aquisição, análise de sinais e algoritmos para processamento destes sinais aquisitados, que será implementada no software supervisor.

\section{\textbf{Manutenção}}
Todo e qualquer meio no qual se pretenda fabricar alguma coisa precisa de meios que permitam a produção (Nepomuceno, 1989), estes meios podem variar desde instrumentos simples, como tesouras, agulhas e novelos de lã, até maquinário industrial de alta complexidade. 
Em qualquer destes casos sempre surgem as questões de desgaste, enguiços, quebras e fraturas, dentre outros possíveis problemas. Portanto, em toda atividade produtiva  há necessidade de manutenção, sem a qual a produção colapsa (Nepomuceno, 1989).
Usualmente classifica-se a manutenção em três formas distintas: corretiva, preventiva e preditiva (Holanda, 2016). 
	A manutenção corretiva, também conhecida por manutenção reativa, diz respeito a prática correspondente a primeira geração das técnicas de manutenção. A manutenção corretiva foi utilizada desde as primeiras industrias até, aproximadamente, o final dos anos 40 do século XX – período em que a indústria era pouco mecanizada e com maquinário comparativamente mais simples. Seu princípio é a utilização do equipamento até sua quebra, sendo apenas então realizado o reparo (Holanda, 2016). 
	
Entretanto, de acordo com Holanda, 2016	

\begin{citacao}

	Este tipo de manutenção baseada no tempo, porém, não elimina a possibilidade de ocorrência de falhas inesperadas no período, visto que a taxa de falha de 		muitas máquinas não é melhorada com a substituição regular de partes gastas. Pelo contrário, frequentemente a confiança nas máquinas recém-trabalhadas é 		reduzida, temporariamente, devido à interferência humana. (Holanda, 2016)

\end{citacao}

Assim, faz-se necessária uma forma de manutenção que particularize cada máquina, de forma a avaliar sistematicamente suas condições. Isto começa a ocorrer nos 70 do século XX, com os avanços da automação e elevação das exigências de confiabilidade e disponibilidade. A este método de manutenção chama-se de manutenção preditiva. (Holanda, 2016).  


\subsection{\textbf{Manutenção preditiva}}

Para Nepomuceno, 1989, p.41 a manutenção preditiva tem por finalidade 

\begin{citacao}

[...]estabelecer, numa instalação industrial qualquer, quais são os parâmetros que devem ser escolhidos em cada tipo de máquina ou equipamento, em função das informações que as alterações de tais parâmetros sobre o estado mecânico de um determinado componente (pistões, dilatação, rolamentos, vasão, particulado, etc.). Em base a tais informações, a análise dos mesmos permitirá que sejam tomadas providências visando evitar estragos de monta ou mesmo situações catastróficas irreversíveis. (Nepomuceno, 1989, p.41)

\end{citacao}

Assim, portanto, “a manutenção preditiva visa realizar manutenção somente quando as instalações precisarem dela” (Slack, 2008, p.645). 
De acordo com Nepomuceno, 1989, quando se estuda a manutenção preditiva esta, em uma análise superficial, parece demasiadamente dispendiosa, há a evidente necessidade de medição e análise de diversos parâmetros e, para tanto, é necessários instrumentos de análise e medição, além de alteração de processos convencionais de medição e controle da condição do maquinário. Entretanto, ainda de acordo com Nepomuceno, 1989, analisando mais detalhadamente esta questão percebe-se que a manutenção preditiva apresenta vantagens que a tornam economicamente viável. Dentre outras, cita-se: 


\begin{itemize}
	\item Reparos se tornam menos custosos que uma quebra de maquinário e consequente interrupção da produção;
	\item A rejeição é diminuída devido ao constante ajuste do equipamento, acarretando, portanto, em menor perda de materiais;
	\item Equipamentos reserva podem, em muitos casos, ser eliminados;
	\item Controle de peças e materiais sobressalentes, diminuindo sensivelmente os custos de estoque;
	\item Controle de componentes mais substituídos bem como de equipamentos que apresentam maior número de defeitos permite o rastreamento da origem dos problemas: se no material, equipamento ou operadores;
	\item Diminuição da ociosidade produtiva acarretada por falhas;
	\item Redução de 15\% a 20\% do custo global em comparação aos métodos clássicos de manutenção, quando computados os custos de peças, materiais e mão-de-obra;
	\item Diminuição sensível, ou mesmo eliminação, da manutenção corretiva;
\end{itemize}

Além destes argumentos Holanda, 2016, chama a atenção para o fato de a manutenção preditiva ter como benefício colateral a capacidade de determinação automática do Tempo Médio Entre Falhas, MTBF (Mean Time Between Failures). Segundo Holanda, 2016

\begin{citacao}
Esse indicador fornece os subsídios para se determinar o tempo mais efetivo em termos de custo para substituir o maquinário, ao invés de continuar a absorver altos custos de manutenção. No momento em que o MTBF atinge o ponto no qual os custos de manutenção e de operação do equipamento excederem os custos da substituição, a máquina deve ser substituída. (Holanda, 2016)
\end{citacao}

Para a condução de um processo de manutenção preditiva existem diferentes técnicas. Holanda (2016) cita como as principais técnicas: termografia, técnica que, através de temperatura, realiza a formação de imagens térmicas para atuação no diagnóstico de falhas; análise de lubrificantes, técnica que verifica fatores que possam ser danosos a operação do equipamento através da análise de viscosidade, acidez, teor da água dentre outros; ultrassom, técnica que permite a detecção de descontinuidades internas através da propagação de ondas através do equipamento; líquido penetrante, técnica que possibilita a verificação de descontinuidades superficiais tais como trincas, poros e dobras; por fim, cita-se a análise de vibrações, técnica que permite o monitoramento do desenvolvimento de falhas em maquinário e peças através de medições de suas assinaturas vibracionais. A manutenção preditiva baseada em análise de vibrações será abordada em maiores detalhes em seções posteriores deste trabalho.


\section{\textbf{Fundamentos e aplicações de rolamentos}}
Rolamentos são dispositivos para transmissão de movimentos rotacionais ou lineares, com finalidade de redução de atrito entre partes móveis (Santos et al., 2007). Rolamentos são compostos, geralmente, de dois anéis, elementos rolantes e uma gaiola, sendo classificados como rolamentos radiais ou axiais, dependendo da direção da carga principal (NSK, 2018). Adicionalmente, os rolamentos são classificados como rolamentos de esferas e rolamentos de rolos, em função do tipo de elemento rolante, e segregados quanto seus propósitos específicos (NSK, 2018).

A Figura 1, abaixo, demonstra os componentes de um rolamento de uma carreira esferas e suas dimensões. 

\begin{figure}[!htb]
\centering
\includegraphics{Figura1}
\caption{a) Componentes de um rolamento. (b) Dimensões do rolamento. Mesquita et al. 2002 apud Santos et al. 2017}
\label{Figura1}
\end{figure}

Segundo a NSK (2018), 

\begin{citacao}
Atualmente, os rolamentos são uma das peças de máquina mais comumente usadas porque seu movimento de rolamento torna quase todos os movimentos mais fáceis e ajudam a reduzir a fricção. Rolamentos têm duas funções-chave: transferem movimento, isto é, suportam e guiam componentes que giram um em relação ao outro; transmitem forças.
\end{citacao}

Neste âmbito justifica-se o estudo dos rolamentos, suas principais características, propriedades e falhas.

\subsection{\textbf{Tipos de rolamentos}}

Os rolamentos, de forma semelhante às rodas, são componentes que podem rolar  e que servem para redução de atrito entre eles e a superfície (Abecom, 2019). Existem distintos tipos de rolamentos, dentre os quais cita-se:

\begin{itemize}
	\item Rolamentos de rolos cilíndricos, constituídos por uma carreira de rolos em uma gaiola; 
	\item Rolamentos de rolos cônicos, contrítudos por pistas de anel interno e externo, cônicas e rolos cônicos
	\item Rolamentos autocompensadores, que possuem duas carreiras de esferas ou rolos com uma pista esférica comum no anel externo e duas pistas no anel interno 
	\item Rolamentos rígidos de esfera, que são rolamentos versáteis de concepção simples e tipo não separáveis
	\item Rolamentos de esfera de contato angular, que possui pistas nos anéis interno e externo
	\item Rolamentos axiais
\end{itemize}

\subsection{\textbf{Processos de fabricação de rolamentos}}

Nos rolamentos rígidos de esferas, modelo comum de rolamentos e objeto de estudo deste trabalho, segrega-se o processo de fabricação em: fabricação dos anéis interno e externo; fabricação das esferas; fabricação das gaiolas e, por fim, montagem (Dos Santos et al., 2017).   

As esferas são feitas a partir de barras de aço cortadas em comprimentos específicos. Um processo de forjamento destes fragmentos de barra confere às peças um formato esférico, mas ainda com rebarbas. É realizado, então, o desbaste destas rebarbas. Após o processo de desbaste as esferas passam por tratamento térmico de têmpera. Por fim, é realizado o polimento. 

Os anéis interno e externo, por sua vez, são criados pelos mesmos processos, a saber: forjamento, torneamento, tratamento térmico, polimento e super-acabamento. No  primeiro processo, forjamento, os lingotes de aço são aquecidos e, através de prensagem, são formados os anéis interno e externo em cada movimento da prensa. Os processos decorrem, portanto, simultaneamente e no mesmo equipamento. Após o forjamento, se dá o processo de torneamento. Neste, o torneamento do anel interno e externo se dá pelos mesmos passos. Primeiramente é torneada uma face lateral, a largura do anel. Em seguida, a outra face lateral é torneada. O próximo passo do processamento do anel é o torneamento do perfil da pista e largura. Simultaneamente é usinado o chanfro do anel. Em sequência ao processo de torneamento, se dá o tratamento térmico dos anéis por tempera, onde estes são aquecidos a temperaturas superiores aos 800ºC, bruscamente resfriados em óleo, reaquecidos em temperaturas de 150ºC e resfriados ao ar. Por fim, há o processo de acabamento onde processos abrasivos e de polimento são utilizados para atingir acabamento fino e exatidão de dimensão das peças (Dos Santos et al., 2017). 

A Figura 2, abaixo, demonstra o fluxo de ordenação destes processos acima descritos.
\begin{figure}[!htb]
\centering
\includegraphics{Figura2}
\caption{a) Processo de fabricação dos anéis internos e externos. Retirado de Dos Santos et al., 2017.}
\label{Figura2}
\end{figure}

As gaiolas, por sua vez, são produzidas com lâminas de aço que recebem a forma de anéis planos através de estampagem, sendo, então, prensadas para receber forma ondulada afim de manter as esferas na posição, feitos os furos para rebites e tratamento térmico da peça. (Dos Santos et al., 2017).
 
Com todas as partes produzidas é realizada, então, a montagem do rolamento. Primeiro, os anéis interno e externo são colocados concentricamente na linha de produção com as esferas preenchendo os espaços entre estes. O tamanho das esferas é calculado a partir do diâmetro da ranhura do anel interno. As esferas, então, são uniformemente organizadas por um divisor e uma metade da gaiola é adicionada com os furos para rebites. Em sequência outra máquina adiciona a metade restante da gaiola, que possui os rebites, e finaliza a montagem da peça. (Santos et al., 2017). 


\subsection{\textbf{Carga e vida útil de rolamentos}}

A vida nominal básica de um rolamento é dada na norma ISO 281:2007, que também específica os métodos de cálculo de classificação de carga dinâmica básica. Em relação a classificação de carga dinâmica básica, C, esta é utilizada para cálculos que envolvem rolamentos dinamicamente tensionados (SKF, 2019). Tal classificação expressa a carga do rolamento que resultará em uma vida nominal básica, L10, de um milhão de revoluções (SKF , 2019). O valor de C é usualmente encontrado nos catálogos referência das fabricantes.   

Para a vida nominal básica temos na ISO 281:2007 que:

\[L_{10}= \left ( \frac{C}{P} \right )^{p}  \]

Onde $L_{10}$ é a vida nominal básica, com 90\% de confiabilidade, expressa em milhões de revoluções; C é a classificação da carga dinâmica básica, em kN; P é a carga dinâmica equivalente do rolamento, em kN; e, por fim, p é o expoente da equação de vida, que é 3 para rolamentos de esferas e 10/3 para rolamentos de rolos (SKF, 2019).

Se a velocidade for constante, por sua vez, torna-se preferível calcular a vida útil em horas de operação (SKF, 2019). Para tanto, utiliza-se :

\[L_{10h}= \frac{10^{6}}{60_{n}}L_{10}\]

Onde $L_{10h}$ é a vida nominal básica, com 90\% de confiabilidade, expressa em horas operacionais e n é a velocidade de rotação, em rotações por minuto. 
Para que se possa calcular a vida nominal básica de um rolamento usando as classificações de cargas dinâmicas básicas é necessário, entretanto, que esta seja convertida, primeiramente, em carga dinâmica equivalente do rolamento. A carga dinâmica equivalente do rolamento, P, é “definida como uma carga hipotética, com magnitude e direção constantes, que atua radialmente em rolamentos radiais e axial e centralmente em rolamentos axiais” (SKF, 2019). Esta carga hipotética teria influência na vida do rolamento igual às cargas reais que o rolamento é submetido.

Para os rolamentos híbridos, por sua vez, as mesmas vidas nominais podem ser utilizadas. Segundo a SKF, 2019

\begin{citacao}
Experiências e testes diversos mostram que, em aplicações de máquinas-ferramentas normais, a vida útil de um rolamento híbrido é significativamente maior do que a vida útil de um rolamento com elementos rolantes de aço. A vida útil estendida de rolamentos híbridos deve-se à dureza, baixa densidade e acabamento superficial dos elementos rolantes. A baixa densidade minimiza a carga interna das forças centrífuga e de inércia, enquanto a maior dureza torna os elementos rolantes menos suscetíveis ao desgaste. Seu acabamento superficial permite que o rolamento otimize os efeitos do lubrificante. (SKF, 2019)
\end{citacao}

Em relação à carga mínima requerida, rolamentos que operem em alta velocidade e/ou estejam sujeitos a acelerações bruscas, forças de inércia dos elementos rolamentos e atrito do lubrificante podem ser prejudiciais aos elementos rolantes e as pistas (SKF, 2019). Assim, para o adequado funcionamento os rolamentos devem estar sujeitos a uma carga mínima de 0,01C para rolamentos de esferas e 0,02C para rolamentos de rolos. 

Por fim, em determinadas aplicações as condições operacionais, tais como a magnitude, sentido das cargas, velocidades, temperaturas e lubrificação são variáveis (SKF, 2019). O cálculo de vida útil em condições operacionais variáveis não pode ser calculado antes da redução do espectro de carga e do ciclo de trabalho. Para tanto, cada nível de carga pode ser acumulado e o espectro de carga reduzido a um histograma de blocos constantes de carga (SKF, 2019).

Para o cálculo de vida útil em condições variáveis, utiliza-se:

\[L_{10}= \frac{1}{ \frac{U_{1}}{L_{10 \,1}} + \frac{U_{2}}{L_{10 \,2}} + \frac{U_{3}}{L_{10 \,3}} + ...}\]

Onde $L_{10}$ é a vida nominal básica, com 90\% de confiabilidade, $L_{10 \,1}$, $L_{10 \,2}$ são as vidas nominais básicas, com 90\% de confiabilidade, em condições constantes 1, 2 e, por fim, $U_{1}$, $U_{2}$ são as frações, que devem totalizar 1, do ciclo de vida sob as condições 1, 2 etc. De acordo com SKF, 2019

\begin{citacao}
Dentro de cada intervalo de trabalho, a carga e as condições operacionais do rolamento podem ter um valor médio constante. O número de horas de operação ou revoluções esperadas de cada intervalo de trabalho que mostram a fração de vida necessária para essa condição de carga específica. Portanto, se ${N_{1}}$ for igual ao número de rotações necessárias na condição de carga ${P_{1}}$, e N for o número esperado de rotações para concluir todos os ciclos de carga variável, então a fração de ciclo ${U_{1} = N_{1} / N}$ é utilizada pela condição de carga ${P_{1}}$, que possui uma vida útil calculada de ${L_{10 \,1}}$. (SKF, 2019)
\end{citacao}


\subsection{\textbf{Falhas em rolamentos}}

Quando há correto manuseio dos rolamentos, usualmente estes podem ser utilizados por longos períodos de tempo antes de surgirem sinais de fadiga. Entretanto, devido a seleção incorreta do rolamento, lubrificação inadequada ou mesmo falhas de manuseio ou de condição de serviço, podem ocorrer falhas prematuras (NTN, 2018).

Ainda que a operação se dê de forma adequada, entretanto, a ação das tensões cíclicas de cisalhamento causa o aparecimento de microfissuras que, em sua maioria, surgem em pontos de pouca resistência, ou onde o material é anisotrópico  ou em pontos onde ocorrem inclusões de materiais não metálicos. Assim, com o tempo, estas microfissuras evoluem para a superfície da pista onde surgirão microtrincas que evoluem gradativamente (Harris, Juvinall e Marshek).

Dentre as falhas mais comuns em rolamentos temos as seguintes: descascamento, travamento, trinca e lascamento, gaiola danificada, desgaste irregular, manchas e asperezas superficiais, oxidação e corrosão, corrosão por atrito, desgaste regular, corrosão eletrolítica, amassados e arranhamentos, deformação, manchas e descorolação.

Descascamento é o processo em que a superfície da pista começa a descascar, logo sem seguida sendo percebidas ondulações. Dentre as possíveis causas para esta falha estão: excesso de carga ou manejo inadequado, montagem inadequada, precisão incorreta no eixo ou alojamento, folga insuficiente, contaminação, oxidação, lubrificação inadequada ou mesmo queda de dureza devido temperaturas altas anormais (NTN, 2018). A Figura 3, abaixo, mostra uma pista com descascamento.

\begin{figure}[!htb]
\centering
\includegraphics{Figura3}
\caption {Pista de rolamento com descascamento. Retirado de NTN, 2018.}
\label{Figura3}
\end{figure}

Travamento, por sua vez, diz respeito ao processo em que o rolamento se aquece e descolora, eventualmente travando. Dentre as possíveis causas para esta falha estão folga insuficiente no rolamento, incluindo-se folgas que diminuem em razão de deformações locais, lubrificação insuficiente ou imprópria, cargas excessivas ou ainda corpos rolantes afinados nas extremidades (NTN, 2018). A Figura 4, abaixo, mostra o dano em um rolamento que travou.

\begin{figure}[!htb]
\centering
\includegraphics{Figura4}
\caption {Dano causado por trava em rolamento. Retirado de NTN, 2018.}
\label{Figura4}
\end{figure}


O aparecimento de trinca e lascamento se dá quando ocorre um descascamento localizado, nisto surgem pequenas trincas ou lascamentos, dentre as possíveis causas para esta falha estão excessivas cargas de choque, excessiva interferência, formação de grandes descascamentos, formação de descascamento por atritos, encostos ou chanfros inadequados ou, ainda, manejo inadequado (NTN, 2018). A Figura 5, abaixo, mostra rolamento com trincas e lascamentos. 

\begin{figure}[!htb]
\centering
\includegraphics{Figura5}
\caption {Rolamento com lasca e trincas. Retirado de NTN, 2018.}
\label{Figura5}
\end{figure}

Em relação a gaiola danificada, esta falha acontece quando os rebites  soltam ou se quebram. Isto resulta na quebra da gaiola. As possíveis causas para esta falha são carga de momento, altas rotações ou excesso de variação de rotação, lubrificação inadequada, impacto com objetos estranhos, vibração excessiva, montagem inadequada ou, mesmo, aumento anormal da temperatura (NTN, 2018). A Figura 6, abaixo, mostra gaiolas danificadas. 

\begin{figure}[!htb]
\centering
\includegraphics{Figura6}
\caption {Gaiolas danificadas. Retirado de NTN, 2018.}
\label{Figura6}
\end{figure}

Padrões de desgaste irregulares ou abrasão, por sua vez, ocorrem pela ação dos corpos rolantes ao longo das superfícies das pistas. Causas para esta falha incluem eixo ou alojamento com precisão insuficiente, instalação imprópria, insuficiência de rigidez do eixo ou alojamento, ou, mesmo, giro do eixo causado pelo excesso de folga interna do rolamento (NTN, 2018). A Figura 7, abaixo, mostra padrões de desgastes nas pistas do rolamento.

\begin{figure}[!htb]
\centering
\includegraphics{Figura7}
\caption {Rolamentos com padrões de desgaste irregulares. Retirado de NTN, 2018.}
\label{Figura7}
\end{figure}

Outra falha comum, manchas e asperezas superficiais, ocorre quando a superfície se torna áspera e se formam pequenos depósitos. As causas incluem lubrificação inadequada, presença de partículas estranhas, afinamento das pontas dos corpos rolantes em razão de um desalinhamento, ruptura da película lubrificante devido excesso de cargas axiais, rugosidade superficial ou, ainda, deslizamento excessivo dos corpos rolantes (NTN, 2018). A Figura 8, abaixo, apresenta rolamentos com manchas e asperezas superficiais.

\begin{figure}[!htb]
\centering
\includegraphics{Figura8}
\caption {Rolamentos com manchas e asperezas. Retirado de NTN, 2018.}
\label{Figura8}
\end{figure}

Oxidação e corrosão é uma falha que ocorre quando a superfície se torna parcial ou totalmente oxidada, e ocasionalmente óxido também surge ao longo das linhas do passo dos elementos rolantes. Causas para esta falha incluem condições inadequadas de armazenamento, embalagem inadequada, óleo protetivo insuficiente, penetração de água ou ácido, dentre outros elementos, ou, ainda, manuseio inadequado (NTN, 2018). A Figura 9, abaixo, mostra um rolamento oxidado e corroído.

\begin{figure}[!htb]
\centering
\includegraphics{Figura9}
\caption {Rolamentos com manchas e asperezas. Retirado de NTN, 2018.}
\label{Figura9}
\end{figure}

Corrosão por atrito é um tipo de desgaste por raspagem corrosiva. Há duas formas de corrosão por atrito: no primeiro, forma-se um pó de óxido sobre as superfícies de contato; no segundo, ocorre afundamentos nas pistas ao longo do passo dos corpos rolantes. As causas para estas falhas incluem ângulo de oscilação pequeno do rolamento, lubrificação insuficiente, cargas variáveis, vibração durante transporte e, ainda, interferência insuficiente (NTN, 2018). A Figura 10, abaixo, mostra corrosão por atrito.

\begin{figure}[!htb]
\centering
\includegraphics{Figura10}
\caption {Rolamentos com manchas e asperezas. Retirado de NTN, 2018.}
\label{Figura10}
\end{figure}

Desgaste, por sua vez, é o processo em que as superfícies se desgastam e produzem uma deformação dimensional. O desgaste frequentemente acompanha rugosidade e riscos. As causas para o desgaste incluem presença de partículas estranhas no lubrificante, lubrificação inadequada e rolos afinados nas pontas. A Figura 11, abaixo, mostra um rolamento com desgaste (NTN, 2018).

\begin{figure}[!htb]
\centering
\includegraphics{Figura11}
\caption {Rolamento com desgaste. Retirado de NTN, 2018.}
\label{Figura11}
\end{figure}

Corrosão eletrolítica diz respeito a formação de afundamentos sobre as pistas. Estes afundamentos transformam-se gradualmente em ondulações. A causa para isto é a corrente elétrica passando através dos corpos rolantes (NTN, 2018). A Figura 12 mostra a corrosão eletrolítica. 

\begin{figure}[!htb]
\centering
\includegraphics{Figura12}
\caption {Corrosão eletrolítica em rolamentos. Retirado de NTN, 2018.}
\label{Figura12}
\end{figure}

Amassados e arranhamentos são: riscos durante a montagem, arranhamentos em razão de objetos estranhos e duros, além de amassados superficiais devido a impactos. As causas são presença de objetos estranhos, penetração por impacto no lado descascado, quedas e choques devido manejo inadequado além de montagem desalinhada (NTN, 2018). A Figura 13 mostra amassados e arranhamentos. 

\begin{figure}[!htb]
\centering
\includegraphics{Figura13}
\caption {Rolamentos com amassados e arranhamentos. Retirado de NTN, 2018.}
\label{Figura13}
\end{figure}

Deformação de deslizamento é acompanhada de superfícies brilhantes ou de superfícies brilhantes ou de superfícies descoloridas no anel interno ou externo, também podendo ocorrer desgaste abrasivo. As causas são interferência insuficiente na seção sem brilho, bucha de montagem não fixada apropriadamente, aumento anormal de temperatura ou, ainda, cargas excessivas (NTN, 2018). A Figura 14 mostra deformação de deslizamento.

\begin{figure}[!htb]
\centering
\includegraphics{Figura14}
\caption {Deformação de deslizamento. NTN, 2018.}
\label{Figura14}
\end{figure}

Por fim, manchas e descoloração demonstram superfícies sem brilho, a superfície está fosca, rugosa ou eventualmente deformada, também estando coberta de pequenas depressões. As causas são infiltração de substancia estranha ou mesmo lubrificação insuficiente (NTN, 2018). A Figura 15 mostra manchas e descoloração em rolamentos.

\begin{figure}[!htb]
\centering
\includegraphics{Figura15}
\caption {Manchas e descoloração em rolamentos. Retirado de NTN, 2018.}
\label{Figura15}
\end{figure}

Ao verificar-se que houve o contato entre a superfície defeituosa de um elemento rolante e outra superfície sem defeito, o choque entre estes elementos, causa um impulso que causa uma excitação tanto na máquina, quanto no rolamento. 

Para detecção das falhas pode-se utilizar análise de vibrações. Para tanto, os conceitos fundamentais de vibrações mecânicas, bem como particularidades de rolamentos, são apresentados.

\section{\textbf{Fundamentos de vibrações mecânicas}}

Uma vez abordada a importância de rolamentos para a indústria e a relevância da manutenção preditiva, aborda-se a questão dos fundamentos da manutenção preditiva de rolamentos baseada em análise de vibrações e suas vantagens. Para que se possa sobre manutenção preditiva baseada em análise de vibrações, entretanto, se faz necessário primeiramente um entendimento relativo ao fenômeno das vibrações mecânicas. 

Segundo Rao, 2008, p.5

\begin{citacao}
A maioria das atividades humanas envolve vibração de uma forma ou outra. Por exemplo, ouvimos porque nossos tímpanos vibram, e vemos porque as ondas de luz sofrem vibração. A respiração está associada à vibração dos pulmões, e andar envolve movimento oscilatório (periódicos de pernas e mãos). Falamos devido ao movimento oscilatório da laringe (e da língua). (Rao, 2008)
\end{citacao}

Ainda, segundo Holanda, 2016, as vibrações mecânicas são fenômenos que acontecem constantemente na natureza e no cotidiano humano. Holanda define vibração mecânica como

\begin{citacao}
\,[...]um tipo de movimento no qual se considera uma massa reduzida a um ponto submetido a uma força. A ação dessa força obriga o ponto a executar um movimento oscilatório. Para que o movimento oscilatório do ponto se constitua numa vibração, ele deverá percorrer uma trajetória denominada trajetória completa ou ciclo, conhecida pelo nome de período de oscilação. (Holanda, 2016)
\end{citacao}

Nepomuceno, 1989, p.229 afirma que as vibrações mecânicas são fenômenos observados quando uma partícula realiza um movimento entorno uma posição de equilíbrio. Assim, quando há troca de energia cinética em energia potencial e vice-versa, ocorre a vibração. 
Para que se compreenda as vibrações mecânicas, há de se abordar alguns conceitos fundamentais. 
Quando um corpo oscila em períodos determinados ao redor de uma posição de equilíbrio, caracteriza-se este movimento como Movimento Harmônico, forma mais simples que a vibração pode se apresentar (Holanda, 2016). Holanda, 2016, apresenta como exemplo disto o movimento do pêndulo de um relógio, que oscila da máxima posição à direita até a máxima posição à esquerda, passando pelo ponto central de equilíbrio. Este movimento é graficamente representado como uma senoide, conforme mostrado na Figura 16, abaixo.

\begin{figure}[!htb]
\centering
\includegraphics{Figura16}
\caption {Representação do movimento harmônico simples através do pêndulo de um relógio. A senóide representa graficamente o movimento. Retirado de Holanda, 2016, p.20}
\label{Figura16}
\end{figure}

Ressalta-se que o tempo que a massa demora para realizar uma oscilação, ou movimento harmônico simples, é chamado de período. Uma oscilação por segundo constitui uma medida de frequência denominada Hertz (Nepomuceno, 1989). 

Em relação ao movimento harmônico simples, ou oscilação, Nepomuceno, 1989, p.233 ressalta que

\begin{citacao}
Nos casos práticos, existem inúmeros osciladores mecânicos que são dispositivos que dão origem a oscilações simples, principalmente às amplitudes pequenas ou então fornecem combinações de oscilações. É importante observar que todo e qualquer sistema que obedeça a Lei de Hooke é apto a oferecer tal tipo de vibrações. (Nepomuceno, 1989, p.233)
\end{citacao}

Importantes conceitos para a compreensão das vibrações mecânicas são a amplitude, frequência e fase. A amplitude, A, se relaciona com a quantidade de energia contida no sinal vibratório, ou seja, a severidade do movimento, e é medida em milímetros (mm). Pode ser tomada em forma de deslocamento, velocidade e aceleração.  A frequência, f, conforme anteriormente mencionado, é medida em Hertz e representa o número de vezes em que ocorre um ciclo completo de movimento. A fase, $\upphi$, representa “o ângulo inicial do argumento da função senoidal que descreve o movimento harmônico” (Holanda, 2016).
 
Para que se mensure os níveis de vibração, utiliza-se parâmetros expressos em termos de deslocamento, velocidade e aceleração (Holanda, 2016). Segundo Kardec e Nascif, 2009, p.244 todos os três parâmetros representam “o quanto o equipamento está vibrando”. A ISO 10816 de 1995 define e fornece a limitação destes parâmetros. 

O Deslocamento é medido em micrometros (\SI{}{\micro\metre}). De acordo com Holanda, 2016, p.21 o deslocamento pode ser medido pelo nível de distanciamento do ponto relativo à sua posição de repouso, sendo a “unidade mais óbvia para se mensurar a vibração, pois é aquela que mais se aproxima da ideia de oscilação em torno de um ponto médio”. O deslocamento é recomendado para medições abaixo de 10Hz (Holanda, 2016). A fórmula abaixo representa o deslocamento, x:

\[x = A \,sen (\omega t + \upphi)\]

A velocidade, por sua vez, é medida em milímetros por segundo ($mm/s$). Segundo Holanda, 2016, o deslocamento implica na existência de uma velocidade que, por sua vez, pode ser variável. A função de velocidade, v:

\[v = A\omega \,cos (\omega t + \upphi)\]

Holanda, 2016, p.21 afirma que a velocidade de vibração é parâmetro “menos representativo para componentes tanto de baixa como de alta frequência, sendo o parâmetro normalmente escolhido para avaliação da severidade de vibração entre 10 Hz e 1000 Hz”.
A aceleração, por sua vez, é medida em metros por segundo ao quadrado ($m/s^{2}$). Se a velocidade for variável, implica-se a existência de uma aceleração no movimento. 

A fórmula que representa a aceleração, a, é:

\[a = -A^{2} \omega \,sen (\omega t + \upphi)\]

Limitando-se o escopo de abrangência, este fenômeno das vibrações, na engenharia, é um item de alta relevância para estudo visto que os efeitos das vibrações podem ser nocivos, ou mesmo catastróficos, ao se considerar maquinário industrial. Fatores como desgaste precoce de elementos mecânicos e falhas por fadiga causadas por movimentos cíclicos podem ser mensurados ao se monitorar a assinatura vibracional de maquinário e equipamento. 

Assim, tem-se as seguintes fórmulas para o cálculo das frequências de defeitos dos vários  elementos que constituem um rolamento. 


FTF (\textit{fundamental train frequency}) indica a frequência de defeito na gaiola. 

\[FTF = \frac{fr}{2} \left ( 1 - \frac{d}{D} \,cos \, \theta \right )  \]

BPFI, indica a frequência de defeito na pista interna (\textit{ball pass frenquency of the inner race})

\[BPFI = \frac{N}{2}f_{r} \left ( 1 + \frac{d}{D} \,cos \, \theta \right )  \]

BPFO se trata do indicativo da frequência de defeito na pista externa (\textit{ball pass frequency of the outer race})

\[BPFO = \frac{N}{2}f_{r} \left ( 1 - \frac{d}{D} \,cos \, \theta \right )  \]

BSF, por sua vez, é o indicativo da frequência de defeito do elemento rolante (\textit{ball spin frequency}).

\[BSF = \frac{D}{2d}f_{r} \Bigg[  1 - \left ( \frac{d}{D} \,cos \, \theta \right ) ^{2}  \Bigg] \]

Onde, nestas fórmulas, temos $f_{r}$, que é a subtração da frequência de rotação da pista interna pela frequência de rotação da pista externa; $D$ é diâmetro primitivo; $d$ é o diâmetro dos elementos rolantes; $N$ é o número de elementos rolantes; e, por fim, $\theta$ é o ângulo de contato.

\subsection{\textbf{Fundamentos de análise de vibrações}}

Esclarecidos os conceitos fundamentais relativos as vibrações mecânicas, faz-se necessário entender como as vibrações podem ser medidas e analisadas sob a ótica do interesse tecnológico na manutenção preditiva. Nepomuceno, 1989, p.206 afirma que a medida e monitoramento da vibração é uma das atividades de manutenção preditiva que apresenta melhores resultados. 
Mesquita et. al, 2002, afirmam que análise destas vibrações pode se dar nos domínios do tempo, da frequência e no domínio conjunto tempo-frequência..
Em relação à análise no domínio do tempo, Segundo Spamer, 2009, p.62 apud Holanda, 2016, p.22 a amplitude de vibração em função do tempo pode ser medida em valores de pico a pico, valor de pico e valor RMS. 
Meola, 2005, afirma que o valor de pico a pico sinaliza a excursão máxima da onda, medida relevante para definição de quando um elemento da máquina atinge valores críticos. 
Já o valor de pico, de acordo com Meola, 2005, indica níveis máximos de sinais transientes, sem considerar a cronologia da onda. 
O valor médio, por sua vez, indica o quanto o sinal está deslocado quanto à sua referência tendo pouca relevância quanto ao monitoramento da vibração. 
Já o valor RMS, ou eficaz, é, de acordo com Meola, 2005, p.9

\begin{citacao}
\,[...]uma das medidas mais importantes de amplitude, uma vez que quantifica a energia que está diretamente ligado ao fenômeno estudado, no caso é um indicativo do poder destrutivo da vibração. É muito utilizado como parâmetro global para determinação de níveis de alerta e alarme usados em procedimentos de manutenção preditiva.
\end{citacao}

A relação entre os níveis de pico, o nível médio, é dada por:

\[Nivel \, M\acute{e}dio = \frac{1}{T} \int_{0}^{T} x(t)dt \]

Já o nível eficaz, ou nível RMS, de um sinal harmônico é dado por:

\[Nivel \, RMS = \sqrt{ \frac{1}{T} \int_{0}^{T} x^{2}(t)dt }  \]

A Figura 17, abaixo, exemplifica a relação estes níveis.

\begin{figure}[!htb]
\centering
\includegraphics{Figura17}
\caption {Representação dos níveis de um sinal harmônico. Retirado de Meola, 2005.}
\label{Figura17}
\end{figure}

De acordo com Tandon, 1999 apud Meola, 2005, os métodos mais comuns de detecção de falhas no domínio do tempo são o nível RMS e o Fator de Crista, que representa a razão entre o valor de pico e o valor RMS de aceleração. O Fator de Crista, FC, considera a variação do pico e do valor RMS. Considerando-se mancais de rolamentos, Meola afirma que a diferença entre estes é de aproximadamente 3dB, dada por:

\[20log_{10} \left ( \frac{Nivel \,Pico}{Nivel \, RMS} \right ) \]

Esta diferença vai aumentando progressivamente com o surgimento de defeitos, até atingir valor aproximado de 18dB quando, devido ao desgaste geral do rolamento, tal diferença volta a diminuir (Meola, 2005). Assim, ao retornar para aproximadamente 3dB a falha é iminente e o rolamento deverá ser trocado. 

A Figura 18, abaixo, demonstra esta relação entre os valores de Pico e RMS mencionado.  

\begin{figure}[!htb]
\centering
\includegraphics{Figura18}
\caption {Relação entre os valores de Pico e RMS com o desgaste progressivo de rolamentos. Retirado de Meola, 2005.}
\label{Figura18}
\end{figure}	

Já a Figura 19, abaixo, demonstra o comportamento do Fator de Crista no tempo com o desgaste do rolamento.

\begin{figure}[!htb]
\centering
\includegraphics{Figura19}
\caption {Fator de Crista no tempo.}
\label{Figura19}
\end{figure}	

Também descrito por Meola, 2005, é o método da curtose. Neste, os momentos estatísticos são utilizados para descrever o comportamento de variáveis aleatórias. A curtose é utilizada na detecção de falhas incipientes em rolamentos. Segundo Mesquita et al., 2002

\begin{citacao}
A Densidade de Probabilidade do sinal de aceleração de um rolamento também pode ser usada para a detecção de falhas em rolamentos. Um rolamento em boas condições possui uma distribuição Gaussiana de aceleração, enquanto que o rolamento defeituoso produz uma distribuição não-gaussiana devido ao aumento no número dos altos níveis de aceleração.
Vários momentos estatísticos podem ser usados para indicar a forma da densidade de probabilidade. Dyer e Stewart4 propuseram o Fator de Curtose, que é o quarto momento estatístico, normalizado em relação ao desvio padrão elevado a quarta potência.
\end{citacao}

A curtose é descrita, portanto, da seguinte forma:

\[K = \frac{ \int_{-\infty}^{+\infty} (x - \overline{x})^{4}p(x)dx } {\sigma^{4}} \]

Onde $p(x)$ é a função de densidade de probabilidade e $\overline{x}$ o valor médio do sinal de vibração 

\subsection{\textbf{Fluroreto de polivinilideno}}

O fluoreto de polivinilideno, PVDF, é um fluoropolímero termoplátisco altamente inerte produzido a partir da polimerização do monômero chamado difluoretino (kabir et al., 2017). Em 1969 um pesquisador japonês, Heiji Kawai, descobriu altíssimos níveis de atividade piezoelétrica, muito superiores a qualquer polímero natural ou sintético conhecido, no fluoreto de polivinilideno (Marutake, 1995). O PVDF é um polímero semicristalino de cadeia longa, proveniente da unidade CH2-CF2. Esta unidade possui grande um grande momento do dipolo elétrico, de aproximadamente 7,56 x 10-30 C.m. Como estas unidades se alinham de maneira ordenada para uma configuração head-tail superior à 90\%, o polímero apresenta um momento líquido do dipolo elétrico incomumente alto (Chatigny e Robb, 1987).

Dentre outras, o interesse no estudo do PVDF se dá por suas propriedades piezo e piroelétricas. 

Piezoeletricidade é a capacidade de certos materiais, altamente polares, de alterar suas dimensões quando expostos a um campo elétrico, ou, inversamente, gerar sinal elétrico quando mecanicamente deformados (Halvorsen, 1986). A piezoeletricidade foi descoberta em 1880 pelos irmãos franceses Jacques e Pierre Curie, que a observaram em cristais de Quartzo (Manbacchi e Cobbold, 2011). Uma de suas primeiras aplicações tecnológicas foi feita por outro francês, Langevin, que desenvolveu um transmissor e receptor de quartzo para sons submarinos por volta de 1917 - o primeiro sonar (Halvorsen, 1986). Segundo Loussert et al., 2013, p.204 a piezoeletricidade se manifesta pela polarização da célula unitária com o aspecto de momento dipolar $\mu = q \,x \, d$, onde $q$ é a carga elétrica do dípolo e $d$ é a distância entre os centroides.  

As propriedades do PVDF, incluindo a piezoeletricidade, são altamente influenciadas pelo seu grau e tipo de estrutura de cristalina. Chatigny e Robb, 1987, fazem uma revisão relacionada ao processo de fabricação de filmes PVDF para obtenção das características desejadas em cada possível estrutura cristalina deste material. 

Existem três formas distintas de estrutura cristalina no PVDF. A mais comum é centrossimétrica e não polar, obtida quando o polímero é resfriado a partir de seu ponto de derretimento, chamada de fase alfa. A deformação de grãos tipo alfa, com o stretching de filmes obtidos através de extrusão em temperaturas abaixo de 80ºC, faz com que as células unitárias se alinhem em planos paralelos, criando uma fase polar chamada de fase beta. Uma terceira configuração possível é a fase gama, que, ainda que polar, é intermediária em termos de centrossimetria entre as configurações de fases alfa e beta (Chatigny e Robb, 1987).

Para obtenção de níveis significativos de atividade piezoelétrica no material, o polímero em fase beta deve exposto a um campo elétrico de 500 a 1000KV/cm em temperaturas de 80 a 110ºC. Este processo é chamado de poling. O nível de atividade piezo depende do tempo de exposição a este campo elétrico, além da força do campo e da temperatura. Quando conduzido adequadamente, este processo provê orientação permanente para os dípolos no polímero (Chatigny e Robb, 1987).

A Figura 20, abaixo, mostra as configurações alfa, gama e beta.

\begin{figure}[!htb]
\centering
\includegraphics{Figura20}
\caption {Transições de fase do PVDF, induzidas por campo elétrico. Fases alfa, gama e beta. As células unitárias acima são mostradas nos planos ab e bc. O momento dípolo de cada cadeia polimérica é mostrado. Retirado de Wan e Bowen, 2017.}
\label{Figura20}
\end{figure}

Chatigny e Robb, 1987, descrevem um processo típico de preparação de um filme piezo pode ser resumido da seguinte forma:

\begin{enumerate}
	\item Extrusão do PVDF em filmes fase alfa; 
	\item Orientação, uniaxial ou biaxial, a 80ºC e uma relação de stretch de 4:5 para fase beta do filme;
	\item Deposição de eletrodos por algum de vários métodos;
	\item \textit{Poling} térmico a 600KV/cm a 100ºC, por aproximadamente trinta minutos.
\end{enumerate}

Durante este processo, os dípolos são alinhados relativos à direção do campo de \textit{poling}. Quando o filme piezo está operando em modo eletromecânico, o filme se alonga e contrai quando a polaridade dos campos se alterna. Quando operando de forma mecânico-elétrica, por sua vez, forças externas aplicadas produzem tensões de compressão e tração (Chatigny e Robb, 1987). 

Os filmes PVDF são flexíveis e leves, produzidos em uma variedade de espessuras e áreas. Algumas das vantagens dos filmes PVDF incluem elevada resistência química até temperaturas elevadas , 

Devemos considerar o filme piezoelétrico como um material dinâmico que desenvolve uma carga elétrica proporcional a mudança no stress mecânico. O piezo não opera em condições estáticas devido ao rápido decaimento da carga induzida, sendo sua constante de tempo determinada pela constante dielétrica do filme e sua resistência interna. 	Uma possível analogia, fornecida por Chatigny e Robb, 1987, é a de uma esponja despejando e absorvendo um fluido, conforme uma pressão externa é aplicada e então liberada. O filme age como uma esponja liberando uma carga elétrica na frequência que a deformação tem lugar. Uma vez que a deformação cessa, nenhuma carga é transferida. 

O PVDF é um material anisotrópico.  Suas propriedades elétricas, mecânicas e eletromecânicas diferem para as excitações elétricas e mecânicas nas diferentes direções. Desta forma, para uma tabulação sistemáticas das propriedades, utilizamos três eixos identificados por números: 1, correspondente ao comprimento; 2, correspondente a altura; e, por fim, 3, correspondente a espessura (Chatigny e Robb, 1987). 

Em relação a polarização do material, podemos dizer que tensões positivas são elásticas, enquanto tensões negativas são compressivas. Ação elétrica positiva é causada por um aumento na polarização e vice-versa. Para os filmes piezo, o eixo de polarização é sempre a espessura, uma vez que há alinhamento do campo nesta direção. O estresse mecânico, de qualquer forma, pode ser aplicado nas três direções. 

Usualmente utiliza-se algumas constantes para caracterizar as atividades de materiais piezoativos. 

A constante de acoplamento $K$, é a habilidade de trocar energia elétrica por energia mecânica e vice-versa. O quadrado da constante $K$ é igual a energia transformada dividida pela entrada total de energia. Assim, $K312$ é igual a energia elétrica transformada causadora de tensão mecânica ao longo do eixo 1, dividida pela energia elétrica total das faces eletrificadas paralelas ao eixo 3. 

A constante de tensão piezoelétrica, $d$, expressa a razão de tensão desenvolvida ao longo de um eixo específico aplicada paralelamente a um eixo específico. 

\[d_{31} = \frac{Tensao \,no \,eixo \,1}{Campo \,aplicado \,no \,eixo \,3} = \frac{m/m}{V/m} = \frac{m}{v} \]

Além disso

\[d_{31} = \frac{Carga \,por \,area \,de \,eletrodo}{Estresse \,aplicado \,no \,eixo \,1} = \frac{C/m^{2}}{N/m^{2}} = \frac{C}{N} \]

Outra constante é a constante de estresse piezoelétrico $g$, que expressa a razão do campo elétrico ao longo de um eixo especifico pelo estresse aplicado ao mesmo ou outro eixo. A constante g também expressa a tensão ao longo de um especifico pela carga elétrica por área unitária de eletrodos (Chattigny e Robb, 1987). Assim:

\[g_{33} = \frac{Campo \,aplicado \,ao \,longo \,do \,eixo \,3}{Estresse \,aplicado \,ao \,longo ,\do \,eixo \,3} = \frac{V/m}{N/m^{2}} = \frac{Vm}{N} \]

Além disso:

\[g_{33} = \frac{Tensao \,no \,eixo \,3}{Carga \,por \,area \,eletronica} = \frac{m/m}{C/m^{2}} = \frac{m^{2}}{C} \]

Também deve-se considerar a constante hidrostática piezoelétrica, $dh$, que representa a razão de carga de curto-circuito por área unitária da superfície dos eletrodos pelo estresse hidrostático aplicado igualmente ao longo dos três eixos. 

A constante piroelétrica, $p$, por sua vez, está ligada a natureza dos transdutores piezoelétricos. Estes absorvem energia térmica, aumentando assim a sua temperatura e induzindo sinais elétricos. Em filmes piezo o sinal de saída é proporcional à taxa de mudança de temperatura em vez de níveis de temperatura. A constante piroelétrica $p$, assim, relaciona a carga por unidade de área dos eletrodos pela unidade de troca de temperatura. Logo, $p = C/m^{2} K$.

\chapter{Materiais e métodos}

Para realização dos experimentos serão utilizados rolamentos de rolos, sendo estas amostras isentas de qualquer tipo de falha. Estas amostras terão registradas as suas assinaturas de vibração, tanto no domínio da frequência, quanto no tempo.

Serão utilizadas amostras com falhas na pista do anel interno, falhas na pista do anel externo e falhas nos elementos rolantes. Serão aquisitados sinais de todas as amostras com defeitos, com a utilização de sensores fabricados com o polímero PVDF, com terminais que permitem a medição da sua diferença de potencial, quando estes estiverem sujeitos a um esforço mecânico.

A Figura 21, abaixo, mostra o sensor de vibração. 

\begin{figure}[!htb]
\centering
\includegraphics{Figura21}
\caption {Sensor de vibração piezoelétrico.}
\label{Figura21}
\end{figure} 

A posição dos sensores indicativos de falhas por aumento de vibração será nos sentidos longitudinal e transversal; para a verificação do sinal, estes serão comparados com os sinais característicos dos rolamentos sem defeitos para determinação da severidade da falha.

O aparato experimental será composto por um motoredutor cedido pela empresa parceira (WEG / Cestari), com eixo motriz usinado e ligado à mancal de rolamento a ser cedido por empresa parceira (NSK). A bancada possuirá um isolador de vibrações (vibra-stop). A Figura 22, abaixo, mostra o projeto da bancada experimental.

\begin{figure}[!htb]
\centering
\includegraphics{Figura22}
\caption {Bancada experimental. Elaborado pelos autores.}
\label{Figura22}
\end{figure} 

Os modelos matemáticos de identificação de frequências características de defeitos serão utilizados para a análise dos sinais no domínio do tempo e da frequência, que terão a finalidade de calibragem dos níveis de alarme. A análise se dará através de nível RMS, Curtose e Transformada de Fourier.  

Por fim, os custos dos sensores fabricados em PVDF serão calculados e comparados com os equipamentos atualmente disponíveis para aferição do custo vs benefício e um procedimento será elaborado para que pequenas empresas possam utilizar desta metodologia para realização de planos de manutenção preditiva, visando com esta ação reduzir a ocorrência de falhas catastróficas, que são um grande risco tanto para a segurança dos operadores de máquinas quanto para saúde financeira das empresas pela parada não programada da produção.

A análise dos resultados obtidos se dará à luz da revisão bibliográfica e comparativamente com os resultados encontrados em outros artigos. 

A Tabela 1 mostra o cronograma de desenvolvimento.


\begin{table}
\centering
\caption{Cronograma de desenvolvimento}
\resizebox{\textwidth}{!}
{\begin{tabular}{|l|l|l|l|l|l|l|}

\hline
\multirow{2}{*}{Atividades }                                                  & \multicolumn{6}{l|}{2019}                                  \\ 
\cline{2-7}
                                                                              & Julho & Agosto & Setembro & Outubro & Novembro & Dezembro  \\ 
\hline
Finalização de revisão bibliográfica~ ~ ~ ~ ~                                 & X     & X      &          &         &          &           \\ 
\hline
Término da montagem da bancada experimental~ ~ ~ ~ ~                          & X     & X      &          &         &          &           \\ 
\hline
Finalização dos testes dos circuitos e sistema de aquisição de dados~ ~ ~ ~ ~ &       & X      & X        & X       &          &           \\ 
\hline
Testes experimentais~ ~ ~ ~ ~~                                                &       & X      & X        & X       &          &           \\ 
\hline
Análise de resultados~ ~ ~ ~ ~                                                &       &        &          & X       &          &           \\ 
\hline
Qualificação~ ~ ~ ~ ~                                                         &       &        &          & X       & X        &           \\ 
\hline
Submissão de artigo~ ~ ~ ~ ~                                                  &       &        &          &         & X        &           \\ 
\hline
Apresentação em Simpósio/Congresso~ ~ ~ ~ ~                                   &       &        &          &         & X        &           \\ 
\hline
Escrita da dissertação~ ~ ~ ~ ~                                               &       &        & X        & X       & X        &           \\ 
\hline
Defesa~ ~ ~ ~ ~~                                                              &       &        &          &         & X        &           \\ 
\hline
Ajustes e entrega da versão final~ ~ ~ ~ ~                                    &       &        &          &         &          & X         \\
\hline
\end{tabular}}
\label{tab:ExemploDeTabela1}
\source{Os autores, 2019}
\end{table}



% ----------------------------------------------------------
% ELEMENTOS PÓS-TEXTUAIS
% ----------------------------------------------------------
\postextual
% ----------------------------------------------------------

% ----------------------------------------------------------
% Referências bibliográficas
% ----------------------------------------------------------

%\bibliography{referencias}


\begin{thebibliography}{9}

\bibitem{kardec2009}
  KARDEC, Alan; NASCIF, Júlio.
  \textbf{Manutenção: Função Estratégica}.
  3.ed. Rio de Janeiro: Qualitymark: Petrobras, 2009.

\bibitem{Nepomuceno1989}
 NEPOMUCENO, Lauro Xavier.
  \textbf{Técnicas de Manutenção Preditiva}.
  Editora Edgard.1 ed. São Paulo: Blücher, 1989. 1 v.

\bibitem{Rao2008}
 RAO, Singiresu S.
  \textbf{Mechanical Vibrations}.
  4.ed. São Paulo: Pearson Prentice Hall, 2008.
  
\bibitem{Slack2008}
 SLACK, Nigel.
  \textbf{Administração da Produção}.
  2.ed. São Paulo, Atlas, 2008.
  

  

\end{thebibliography}


% ----------------------------------------------------------
% Glossário
% ----------------------------------------------------------
%
% Consulte o manual da classe abntex2 para orientações sobre o glossário.
%
%\glossary

% ----------------------------------------------------------
% Apêndices
% ----------------------------------------------------------


% ---


% ----------------------------------------------------------
% Anexos
% ----------------------------------------------------------

% ---
% Inicia os anexos
% ---
\begin{anexosenv}

% Imprime uma página indicando o início dos anexos
%\partanexos

\end{anexosenv}





\end{document}
